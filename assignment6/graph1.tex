\documentclass{article}
\usepackage{amsthm}
\usepackage{amsmath}

\usepackage{tikz}

\usetikzlibrary{positioning, arrows,automata}
\usetikzlibrary{shapes}
\usepackage{dot2texi}
\begin{document}

\section {K(5)}

\begin{tikzpicture}
\tikzstyle{every state}=[shape=circle, draw=black!50, very thick, fill=blue!10];

\begin{dot2tex}[neato,mathmode]
  graph G {
    node[style="state"];
    a_1 -- a_2
    a_1 -- a_3
    a_1 -- a_4
    a_1 -- a_5

    a_2 -- a_3
    a_2 -- a_4
    a_2 -- a_5
    a_3 -- a_4
    a_3 -- a_5

    a_4 -- a_5
  }
\end{dot2tex}
\end{tikzpicture}\\

\section {K(3,3)}
\begin{tikzpicture}
\tikzstyle{every state}=[shape=circle, draw=black!50, very thick, fill=blue!10];
  
\begin{dot2tex}[styleonly,codeonly, mathmode]
  graph G {
    
    node[style="state"];
    a_1 -- a_4;
    a_1 -- a_5;
    a_1 -- a_6;
    

    a_2 -- a_4;
    a_2 -- a_5;
    a_2 -- a_6;

    a_3 -- a_4;
    a_3 -- a_5;
    a_3 -- a_6;
    
  }
\end{dot2tex}
\end{tikzpicture}

\section {Petersen's Graph}
\begin{tikzpicture}
\tikzstyle{every state}=[shape=circle, draw=black!50, very thick, fill=blue!10];

\begin{dot2tex}[circo,mathmode]
  graph G {
    node[style="state"];
      A -- B;
      A -- E;
      A -- F;
      B -- G;
      B -- C;
      C -- H;
      C -- D;
      D -- E;
      D -- I;
      E -- J;
      F -- H;
      F -- I;
      G -- I;
      G -- J;
      H -- J;
  }
\end{dot2tex}
\end{tikzpicture}

\end{document}
