\documentclass{article}
\usepackage{amsmath}
\usepackage{geometry}
\geometry{top=0.5in}
\title{Arithmetic Progression}
\date{}
\begin{document}
\pagenumbering{gobble}
\maketitle
\date{}
\section*{Definition}
An \textbf{arithmetic progression (AP) or arithmetic sequence} is a sequence of numbers such that the difference between the consecutive terms is constant. For instance, the sequence 3,6,9,12,15 … is an arithmetic progression with common difference of 3.\newline
If the initial term of an arithmetic progression is ${a_1}$ and the common difference of successive members is $d$, then the $n$th term of the sequence ${a_n}$ is given by:
\begin{equation*}
a_n = a_1 + (n - 1)d,
\end{equation*}
and in general
\begin{equation*}
a_n = a_m + (n - m	)d,
\end{equation*}
\section*{Sum}	
The sum of the members of a finite arithmetic progression is called an \textbf{arithmetic series}. For example, consider the sum:
3 + 6 + 9 + 12 + 15 + 18.
This sum can be found quickly by taking the number n of terms being added (here 6), multiplying by the sum of the first and last number in the progression (here 3 + 18 = 21), and dividing by 2:\newline
$\frac{n(a_1 + a_n)}{2}$\newline
In the case above, this gives the equation:\newline
3 + 6 + 9 + 12 + 15 + 18 = $\frac{6(3 + 18)}{2} = \frac{6 \times 21}{2} =$ 40.
\section*{Derivation}
To derive the above formula, begin by expressing the arithmetic series in two different ways:
\begin{align*}
 S_n &= a_1 + (a_1+d) + (a_1+2d) + \cdots + (a_1+(n-2)d) + (a_1+(n-1)d)\\
 S_n &=(a_n-(n-1)d)+(a_n-(n-2)d)+\cdots+(a_n-2d)+(a_n-d)+a_n
\end{align*}
Adding both sides of the two equations, all terms involving $d$ cancel:
\begin{equation*}
\ 2S_n=n(a_1 + a_n)
\end{equation*}
Dividing both sides by 2 produces a common form of the equation:
\begin{equation*}
 S_n=\frac{n}{2}( a_1 + a_n)
 \end{equation*}
An alternate form results from re-inserting the substitution: $ a_n = a_1 + (n-1)d\\ $
\begin{gather*}
S_n=\frac{n}{2}[ 2a_1 + (n-1)d]
\end{gather*}

\newpage

\begin{center}
\vspace{100pt}
{\LARGE Geometric Progression}
\end{center}
\section*{Definition}
 A \textbf{geometric progression}, also known as a \textbf{geometric sequence}, is a sequence of numbers where each term after the first is found by multiplying the previous one by a fixed, non-zero number called the \textbf{common ratio}.\newline
For example, the sequence 2, 4, 8, 16, ... is a geometric progression with common ratio 2. Similarly 20, 10, 5, 2.5, 1.25, ... is a geometric sequence with common ratio 1/2.\newline
Examples of a geometric sequence are powers $r^k$ of a fixed number $r$, such as $2^k$ and $3^k$. The general form of a geometric sequence is\\
$a,\ ar,\ ar^2,\ ar^3,\ ar^4,\ \ldots$\\
where $r \ne$ 0 is the common ratio and $a$ is a scale factor, equal to the sequence's start value.\newline
The $n$-th term of a geometric sequence with initial value $a$ and common ratio $r$ is given by\\
$a_n = a\,r^{n-1}.$\\
Such a geometric sequence also follows the recursive relation\\
$a_n = r\,a_{n-1}$ for every integer $n\geq 1$.
\section*{Sum}
The sum of the members of a geometric progression is called an \textbf{geometric series}. For example, consider the sum:
2 + 4 + 8 + 16 + 32.
The sum can be found out by\newline
$\frac{a(1-r^m)}{1-r}$ where a is the first term (here 2), m is the number of terms (here 5), and r is the common ratio (here 2).
In the example above, this gives:
2 + 4 + 8 + 16 + 32 = $\frac{2(1-2^5)}{1-2} = \frac{-62}{-1}$ = 62.	
\end{document}