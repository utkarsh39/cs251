\documentclass{article}
\usepackage{amsmath}
\title{Arithmetic Progression}
\date{}
\begin{document}
\maketitle
\date{}
An \textbf{arithmetic progression (AP) or arithmetic sequence} is a sequence of numbers such that the difference between the consecutive terms is constant. For instance, the sequence 3,6,9,12,15 … is an arithmetic progression with common difference of 3.\newline
If the initial term of an arithmetic progression is ${a_1}$ and the common difference of successive members is $d$, then the $n$th term of the sequence ${a_n}$ is given by:
\begin{equation}
a_n = a_1 + (n - 1)d,
\end{equation}
and in general
\begin{equation}
a_n = a_m + (n - m	)d,
\end{equation}
\section*{Sum}	
The sum of the members of a finite arithmetic progression is called an \textbf{arithmetic series}.
The sum of the members of a finite arithmetic progression is called an arithmetic series. For example, consider the sum:
3 + 6 + 9 + 12 + 15 + 18.
This sum can be found quickly by taking the number n of terms being added (here 6), multiplying by the sum of the first and last number in the progression (here 3 + 18 = 21), and dividing by 2:\newline
%\vspace{100pt}
$\frac{n(a_1 + a_n)}{2}$\newline
In the case above, this gives the equation:\newline
3 + 6 + 9 + 12 + 15 + 18 = $\frac{6(3 + 18)}{2} = \frac{6 \times 21}{2} =$ 40.
\section*{Derivation}
\end{document}